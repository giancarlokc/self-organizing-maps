\documentclass[journal]{IEEEtran}
\usepackage[brazil]{babel}
\usepackage[utf8]{inputenc}
\usepackage{blindtext}
\usepackage{graphicx}
\usepackage{hyperref}

\hyphenation{op-tical net-works semi-conduc-tor}

\begin{document}
\title{Musical Genre Clustering With Self Organizing Maps}
\author{Giancarlo~Klemm}%

% Title %
\maketitle

% Abstract %
\begin{abstract}
%\boldmath
\blindtext[1]
\end{abstract}

% Key words %
\begin{IEEEkeywords}
IEEEtran, journal, \LaTeX, paper, template.
\end{IEEEkeywords}

\section{Introduction}
\blindtext

	\subsection{Subsection Heading Here}
	\blindtext
	
\section{Previous Work}
\blindtext

\section{Mapa de Kahonen}
\blindtext

\section{Conjunto de dados}
O conjunto de dados utilizado é a coleção de músicas Marsyas GTZAN \href{https://github.com/JustGlowing/minisom}{https://github.com/JustGlowing/minisom}. Este conjunto foi coletado de 2000-2001 de diversas fontes e contêm 1000 segmentos de 30 segundos cada. Há 10 gêneros diferentes com 100 músicas para gênero.

\section{Extração das features}
\blindtext

\section{Implementação do mapa de Kahonen}
A implementação utilizada para o mapa de Kohenen foi a Minisom, que pode ser encontrada em \href{https://github.com/JustGlowing/minisom}{https://github.com/JustGlowing/minisom}. Sobre essa implementação:
\begin{center}
``Minisom é uma implementação minimalista baseada em Numpy de um mapa de Kahonen. Um mapa de Kahonen é um tipo de rede neural capaz de converter relações complexas não-lineares entre itens em relaçoes geométricas simples.''
\end{center}

\section{Metodologia}
Para os testes, redes de tamanho 7x7 até 15x15 foram utilizadas. Com $\sigma=3.0$, taxa de aprendizagem 0.5 e com método de treinamento aleatório usando 100 iterações.

Os gêneros utilizados para os testes foram:
\begin{itemize}
	\item Clássico
	\item Eletrônico
	\item Folk
	\item Rock Clássico
	\item Reggae
\end{itemize}

Para cada gênero, 100 músicas foram utilizadas. Vários testes foram feitos usando permutações diferentes das features disponíveis e gêneros musicais para fazer uma análise sobre quais features são melhores para o conjunto de dados utilizado.

\section{Resultados}
\blindtext

\section{Conclusão}
\blindtext

\ifCLASSOPTIONcaptionsoff
  \newpage
\fi

\begin{thebibliography}{1}

\bibitem{IEEEhowto:kopka}
H.~Kopka and P.~W. Daly, \emph{A Guide to \LaTeX}, 3rd~ed.\hskip 1em plus
  0.5em minus 0.4em\relax Harlow, England: Addison-Wesley, 1999.
  
\bibitem{IEEEhowto:kopka}
G.~Klemm, R.~Valle, T.~Lima, \emph{Automatic Support for Digital Mashups}, 2014.

\end{thebibliography}

\end{document}


